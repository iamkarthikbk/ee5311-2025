%--------------------
% Packages
% -------------------
\documentclass[12pt,a4paper]{article}
% \usepackage[a4paper, left=15mm, right=15mm, top=15mm, bottom=15mm]{geometry}
\usepackage[utf8x]{inputenc}
\usepackage[T1]{fontenc}
%\usepackage{gentium}
\usepackage{mathptmx} % Use Times Font
% \usepackage[siunitx]{circuitikz} % for circuit schematics
\usepackage{siunitx}
\usepackage{amsmath} % for the equation* environment
\usepackage{graphicx}
\usepackage{pgfplots}
\pgfplotsset{compat=1.14}
\usepackage{float}

\usepackage[pdftex]{graphicx} % Required for including pictures % clashes with circuitikz
% \usepackage[swedish]{babel} % Swedish translations
\usepackage[pdftex,linkcolor=black,pdfborder={0 0 0}]{hyperref} % Format links for pdf
\usepackage{calc} % To reset the counter in the document after title page
\usepackage{enumitem} % Includes lists

\frenchspacing % No double spacing between sentences
\linespread{1.2} % Set linespace
\usepackage[a4paper, lmargin=0.08\paperwidth, rmargin=0.08\paperwidth, tmargin=0.08\paperheight, bmargin=0.08\paperheight]{geometry} %margins
%\usepackage{parskip}
% \usepackage[all]{nowidow} % Tries to remove widows
\usepackage[protrusion=true,expansion=true]{microtype} % Improves typography, load after fontpackage is selected

\usepackage[inkscapelatex=false]{svg}
\graphicspath{ {./media/} }

% \pagecolor{black}
% \color{white}

\usepackage{setspace}


%-----------------------
% Set pdf information and add title, fill in the fields
%-----------------------
\hypersetup{ 	
pdfsubject = {},
pdftitle = {ee5311-2025-ee24s053-pwc-report-tut4},
pdfauthor = {Karthik B K <ee24s053@smail.iitm.ac.in>}
}

%-----------------------
% Begin document
%-----------------------
\begin{document}

\title{EE5311 \\ Report of Practical Work Conducted for Tutorial 04}
\author{Karthik B K ee24s053}
\maketitle

\section{Experiment 01}
\subsection{Calculations}
\noindent No manual calculations.

\subsection{Schematics}
\noindent We draw the following schematics for a seven-stage ring oscilator using \emph{xschem}.
\begin{center}
\includegraphics[width=0.99\linewidth]{tut4/expt1/ro7.sch.png}
\end{center}

\subsection{Measurements}
\noindent We perform a transient simulation with the setup shown in Fig 1 and obtain the following plots.
\begin{center}
\begin{tabular}{cc}
     \includegraphics[width=0.47\linewidth]{tut4/reports/media/expt1_freq.png} &
     \includegraphics[width=0.47\linewidth]{tut4/reports/media/expt1_prd.png} \\
     Fig 2a: Frequency of a 7-stage RO & Fig 2b: Period of a 7-stage RO
\end{tabular}
\end{center}
\noindent We obtain the following values for period and frequency.
\begin{verbatim}
    When Vdd = 1.8 V,
        period = 426 ps
        freq = 1/period = 2.34 GHz
\end{verbatim}

\section{Experiment 02}
\subsection{Calculations}
\noindent No manual calculations.

\subsection{Schematics}
\noindent We draw the following schematics for a seven-stage ring oscilator using \emph{xschem}.
\begin{center}
\includegraphics[width=0.99\linewidth]{tut4/expt2/ro9.sch.png}
\end{center}

\subsection{Measurements}
\noindent We perform a transient simulation with the setup shown in Fig 1 and obtain the following plots.
\begin{center}
\begin{tabular}{cc}
     \includegraphics[width=0.47\linewidth]{tut4/reports/media/expt2_freq.png} &
     \includegraphics[width=0.47\linewidth]{tut4/reports/media/expt2_prd.png} \\
     Fig 2a: Frequency of a 7-stage RO & Fig 2b: Period of a 7-stage RO
\end{tabular}
\end{center}
\noindent We obtain the following values for period and frequency.
\begin{verbatim}
    When Vdd = 1.8 V,
        period = 547 ps
        freq = 1/period = 1.82 GHz
\end{verbatim}

\section{Declarations}
\begin{enumerate}
    \item I have publicly hosted this work on GitHub to help reproduce all of my results. The same can be accessed through the following \href{https://github.com/iamkarthikbk/ee5311-2025}{\underline{link}}.
\end{enumerate}

\end{document}
